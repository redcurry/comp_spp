\documentclass[12pt]{letter}

\usepackage{graphicx}

\addtolength{\topmargin}{-0.5in}
\addtolength{\textheight}{0.5in}

\address{Carlos Anderson \\
  203 Natural Sciences Building \\
  Michigan State University \\
  East Lansing, MI 48824 \\
  Telephone: 517-432-3485 \\
  E-mail: carlosja@msu.edu}

\begin{document}
\begin{letter}{The Editors of \emph{Current Biology} \\
  Cell Press \\
  600 Technology Square, 5th floor \\
  Cambridge, MA 02139}

\opening{Dear Editors:}

We would like to submit to \emph{Current Biology} our manuscript
\textbf{Compensatory adaptation causes rapid incipient speciation},
written by Carlos J. R. Anderson and Barry L. Williams,
with the following abstract:



\begin{quote}
{\small\textbf{%
Compensatory mutations suppress the effects of deleterious mutations
without being beneficial alone.
%
Epistatic interactions among compensatory mutations that have
evolved in separate populations may form an intrinsic
postzygotic isolating barrier (i.e., hybrid inviability or sterility),
leading to biological speciation.
%
Indeed, about ten percent of genetic differences between species
comprise compensatory mutations.
%
Using populations of digital organisms, we show that
compensatory adaptation caused more rapid and stronger
postzygotic isolation than in populations evolved through drift.
%
Surprisingly, the strength of this isolation was independent
of the effect size of the original deleterious mutations.
%
We also find that both deleterious and compensatory mutations
contribute equally to reproductive isolation.
%
Our results suggest that compensatory adaptation may be
an important genetic mechanism of speciation,
and supports the view that intrinsic postzygotic isolation
can stem from intrinsic genetic mechanisms.
}}
\end{quote}



In this study, we investigate a new genetic mechanism
that may lead to speciation (i.e., the evolution of reproductive isolation).
%
We address whether compensatory adaptation
can lead to reproductive isolation between populations.
%
Compensatory adaptation is the process in which populations recover
from accumulated deleterious mutations by secondary mutations.
%
We addressed this question using Avida, an artificial life software
designed to study general evolutionary processes.
%
Our work yielded three key findings:
(1) compensatory adaptation can lead to reproductive isolation,
(2) the degree of reproductive isolation was not strongly correlated
to the effect size of deleterious mutations, and
(3) individual compensatory mutations partly caused reproductive isolation.
%
Our findings support the recent idea that intrinsic reproductive isolation
(i.e., hybrid inviability and sterility) derives from adaptation
to the internal, genetic environment (e.g., compensatory adaptation).
%
In addition, our work suggests that compensatory adaptation
may be an important genetic mechanism by which speciation
involving small populations (e.g., founder and peripatric speciation) occurs.



Thank you for your consideration.



% Trick to insert a digital signature
\closing{Sincerely, \\ $ $ \\    % Forces a vertical space
\fromsig{\includegraphics[width=2in]{signature.pdf}} \\
\fromname{Carlos Anderson}}

\end{letter}
\end{document}
