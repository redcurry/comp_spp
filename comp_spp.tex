\documentclass[12pt]{article}

\usepackage{upgreek}

\hyphenation{a-mong post-zy-got-ic wheth-er ac-cu-mu-lat-ed}

\title{Compensatory adaptation causes rapid incipient speciation}
\author{Carlos J. R. Anderson \and Barry L. Williams}

\begin{document}

\maketitle

\begin{abstract}
Compensatory mutations suppress the effects of deleterious mutations
without being beneficial alone.
%
Epistatic interactions among compensatory mutations that have
evolved in separate populations may form an intrinsic
postzygotic isolating barrier (i.e., hybrid inviability or sterility),
leading to biological speciation.
%
Indeed, about ten percent of genetic differences between species
comprise compensatory mutations.
%
Using populations of digital organisms, we show that
compensatory adaptation caused more rapid and stronger
postzygotic isolation than in
populations evolved through drift.
%
Surprisingly, the strength of this isolation was independent
of the effect size of the original deleterious mutations.
%
We also find that
both deleterious and compensatory mutations
contribute equally to reproductive isolation.
%
Our results suggest that compensatory adaptation may be
an important genetic mechanism of speciation,
and supports the view that intrinsic postzygotic isolation
can stem from intrinsic genetic mechanisms.
\end{abstract}



\section*{Introduction}

Biological speciation is caused by the evolution of reproductive isolating
barriers that prevent (or could prevent) gene flow between populations
\cite{coy04}.
%
One such barrier that is well-studied is hybrid sterility or inviability
due to developmental, physiological, or behavioral abnormalities
regardless of the environment (`intrinsic postzygotic isolation').
%
Intrinsic postzygotic isolation is believed to commonly evolve
through genetic incompatibilities: negative epistatic interactions
among species-specific alleles inherited by the hybrids \cite{pre10}.
%
These species-specific alleles are thought to have been driven to fixation
in each population mostly by natural selection,
but the mechanisms of selection are often unknown \cite{sch09}.
%
In order to understand the selection pressures that caused
the fixation of alleles involved in hybrid incompatibilities,
one must identify such alleles as well as their function
and infer the mechanisms of selection \cite{sch09,pre10,nos11}.

% (evolution of RI involves conflict resolution b/w genetic elements)
% (Schluter 2009)

Due to ever-improving methods in genetic mapping and molecular genetics,
many `speciation genes' and their functions have been identified
\cite{noo06,mah11}.
%
(Examples/discussion on these genes; if known, what those genes are,
what they're involved in, their names, species where they were found.)
%Article: Adaptive evolution drives divergence of a hybrid incompatibility gene between two species of Drosophila.
%
Therefore, it has been suggested that intrinsic postzygotic isolation
may be caused mainly by adaptation to the internal, genetic environment
rather than the external, ecological environment \cite{pha09,pre10}.


An interesting class of adaptation to the genetic environment
is compensatory adaptation, in which secondary mutations
compensate for the effects of deleterious mutations
\cite{har96,bur99,moo00,lev00,mai02,est03,est11}.
%
Deleterious mutations may accumulate in a population
through genetic drift, hitchhiking with beneficial mutations,
transient environmental changes, or spread of selfish genetic elements
\cite{pre10}. % Need citations for other mechanisms
%
Because stabilizing selection acts to maintain phenotypic function
via compensatory adaptation, only genotypic, not phenotypic, change occurs
\cite{har96}.
%
As a result, populations that undergo compensatory adaptation
in identical environments will diverge genetically, not phenotypically.
%
If compensatory adaptation can lead to genetic divergence,
independent compensatory adaptation may contribute to
intrinsic postzygotic isolation \cite{orr01,kon02,kul04,lan07,sch09b,pre10}.
%
Deleterious and compensatory mutations accumulated in different populations
may form DMIs in hybrids \cite{har96}, leading to hybrid inviability
and potentially to complete postzygotic isolation.



Whether compensatory adaptation can lead to postzygotic isolation
remains to be tested experimentally.
%
Here we perform such an experiment to answer the following questions:
(1)~does postzygotic isolation evolve from independent compensatory adaptation?
(2)~what is the strength of genetic incompatibilities formed?
and (3)~what is the relative contribution of compensatory mutations
and deleterious mutations to postzygotic isolation?
%
Answering these questions requires that we identify
both deleterious and compensatory alleles,
which involves genetic manipulations that test the allelic
effects of each type of mutation.
%
For example, compensatory mutations must not be beneficial
in the absence of the deleterious mutations they compensate,
and thus they must be tested on their own.
%
Such genetic manipulations, however, are difficult even in
systems where genetic tools have been greatly developed.



We conducted our experiments using the artificial life system
Avida \cite{ofr04}, which has been used previously to study various questions
in evolution \cite{len99,len03,cho04,mis06,ele07,ele08,mis10}.
%
Digital organisms in Avida consist of a genome:
a single (haploid) sequence of instructions that encodes
their ability to replicate and perform computational functions.
%
Variation among organisms in their ability to perform functions
and in their replication efficiency
arises via random mutations and recombination.
%
Organisms that evolve the ability to perform functions
are effectively rewarded with a greater replication rate.
%
Thus, inheritance, variation, and differential reproduction
in digital organisms allow them to evolve
via natural selection and genetic drift.
%
An organism's fitness was calculated based on the rewards
obtained by performing tasks and its replication efficiency;
organisms that could not replicate (i.e., sterile)
were given a fitness of zero.
%
Avida has enabled research on evolving genetic systems
that would have been difficult in natural systems \cite{ada06},
and the similarities between digital and biological organisms
in many evolutionary phenomena have been remarkable \cite{wil02,ada06}.
%
Avida enhances the benefits of microbial systems
(i.e., short generation times, lots of replication,
easy manipulation and storage of genomes),
while being a true instance of evolution of digital genetic systems,
where the generality of evolutionary principles can be tested
\cite{len99,ele08,mis06}.
%
Specific to this study, Avida allowed us to easily insert deleterious mutations
of various effect sizes, carry out thousands of hybridizations,
and individually identify compensatory mutations.



In summary, we isolated mutants with deleterious mutations
from a well-adapted ancestor, and we allowed those mutants to evolve
in replicate for thousands of generations.
%
We then hybridized compensated populations to test for postzygotic isolation,
and we identified individual compensatory mutations to determine
their strength and contribution to postzygotic isolation.
%
We found that (1) postzygotic isolation occured between compensated populations,
(2) the strength of incompatibility among compensatory mutations
was greater than among neutral mutations, and
(3) compensatory mutations contributed as much as deleterious mutation
to postzygotic isolation.
%
Our results suggest that compensatory adaptation may be an important
mechanism for which hybrid incompatibilities, and therefore
intrinsic postzygotic isolation, evolve.
%
Because we used a non-specific genetic system to test our hypothesis,
our results are relevant to biological organisms and motivate
future tests in biological organisms.



\section*{Results}

Starting with a pair of haploid, sexually-reproducing digital organisms,
we allowed three replicate populations to adapt to a diverse environment
for 1,000,000 updates ($\sim$250,000 generations).
%
(An update is the standard unit of time in Avida.)
%
We used the most common genotype of each adapted population
as an independent ancestor for all subsequent experiments.
%
From the ancestors, we isolated 472 mutants
with 1-5 random irreversible deleterious mutations
whose combined fitness effect was either
small ($\Delta W$~=~0.01-0.1) or large ($\Delta W$~=~0.1-0.9).
%
As a control, we also isolated 74 neutral mutants ($\Delta W$~=~0.0)
with the same range of deleterious mutations per genome as those
in the small-effect and large-effect treatments (Table~\ref{tbl1}).
%
We then allowed populations founded by each mutant (including the controls)
to evolve for 25,000 updates ($\sim$6,000 generations)
in identical environmental conditions as their ancestor.
%
We regarded a population as compensated if its most common genotype
(1)~had a fitness at least equal to that that of its ancestor,
(2)~did not acquire mutations that were beneficial on their own, and
(3)~did not acquire mutations that were deleterious when they first appeared.



\subsection*{Reproductive isolation via compensation is rapid}

We performed two types of hybridizations:
(1)~between compensated (or control) populations and their ancestor (`AC')
and (2)~between pairs of compensated (or control) populations (`CC').
%
The first type of hybridization corresponds to the scenario where
a population splits off from an ancestral population,
whereas the second type corresponds to the scenario where
a population splits into two or more subpopulations.
%
All hybrids were generated after completion
of the compensatory adaptation experiments.
%
We found that the mean fitness of hybrids for both hybridization types
was lower than that of hybrids from control populations
(in Fig.~\ref{fig1}, compare `Control' hybrids with `Del.~+~Comp.' hybrids).
%
Surprisingly, whether populations compensated for
small- or large-effect deleterious mutations
did not have a significant effect on mean hybrid fitness
(in Fig.~\ref{fig1}, the 95\% bootstrap confidence intervals overlap for
`Small-effect Del.~+~Comp.' and `Large-effect Del.~+~Comp.').
%
These findings suggest that intrinsic postzygotic isolation
can develop faster during compensatory adaptation than during genetic drift,
regardless of the fitness effect size of the initial deleterious mutations.



\subsection*{Compensatory adaptation forms strong hybrid incompatibilities}

If hybrids between compensated genotypes
have stronger genetic incompatibilities
than hybrids between control genotypes,
then the rate at which hybrid fitness decays
with the number of inherited mutations should be greater
for hybrids between compensated genotypes
than for hybrids between control genotypes.
%
In other words, the slope of the line relating
the number of mutations in hybrids
with the hybrid fitness should be greater
for hybrids between compensated genotypes
than for hybrids between control genotypes.
%
Before we performed this test,
we generated a new set of 250 control genotypes
because our experimental control populations
acquired only 0-1 mutations after 25,000 updates.
%
To generate the new control genotypes,
we introduced 2-10 neutral mutations into ancestral genotypes,
which was within the range of number of mutations in compensated genotypes.
%
We then fit a least squares linear relationship between
the mean number of mutations in hybrids
and the hybrid fitness for each treatment
(i.e., neutral-, small-, and large-effect for both AC and CC hybrids).
%
We found that the slope of this linear relationship
was significantly greater for hybrids between compensated genotypes
than for hybrids between control genotypes (Fig.~\ref{fig2}B),
but was not significantly greater for AC hybrids (Fig.~\ref{fig2}A).
%
Our findings suggest that genetic incompatibilities
involving deleterious and compensatory mutations in hybrids
were stronger than those among neutral mutations.



\subsection*{Both deleterious and compensatory mutations
  contribute to reproductive isolation}

Although we have shown that genetic interactions
involving deleterious and compensatory mutations
may form strong hybrid incompatibilities rapidly,
we have yet to quantify the relative contributions
of each type of mutation to the strength of genetic incompatibilities.
%
At first glance, Fig.~\ref{fig1}
may suggest that compensatory mutations
contributed little to reproductive isolation
because hybrids between genotypes before compensation (`Del.~only')
had fitnesses as low as hybrids after compensation (`Del.~+~Comp.').
%
However, hybrids between compensated genotypes
in which deleterious mutations were reverted to the ancestral state
also had low fitnesses
(in Fig.~\ref{fig1}, compare `Comp.~only' hybrids to `Del.~+~Comp.' hybrids),
suggesting that compensatory mutations
also contributed to postzygotic isolation.
%
To establish more directly the extent
to which deleterious and compensatory mutations
contributed to postzygotic isolation,
we created genotypes with different proportions
of deleterious and compensatory mutations
and measured their fitness.
%
We created such genotypes by taking contiguous regions of different sizes
from an evolved parental genotype and substituting them
into the corresponding region in the ancestral genotype.
%
For each new genotype, we first measured its fitness and then calculated
the proportion of deleterious and compensatory mutations
relative to the parental genotype.
%
We found that, unless genotypes contained either
all or none of their parent's deleterious and compensatory mutations,
their fitness was low (Fig.~\ref{fig3}),
suggesting that both deleterious and compensatory mutations
are important in postzygotic reproductive isolation.



\section*{Discussion}

Most genes involved in intrinsic postzygotic isolation,
i.e., hybrid sterility or inviability
due to developmental or physiological abnormalities,
show strong signatures of positive selection \cite{pre10}.
%
Surprisingly, many of these genes
are not adaptations to the external, ecological environment
but to an impaired internal, genetic environment
(see Table~\ref{tbl1} in \cite{pre10}).
%
An example of adaptation to an impaired genetic environment
is adaptive compensation of deleterious mutations
\cite{har96,bur99,moo00,lev00,mai02,est03,est11}.
%
Populations undergoing independent compensatory adaptation
will diverge genetically
and may form Dob\-zhan\-sky-Mul\-ler incompatibilities,
causing intrinsic postzygotic isolation
\cite{orr01,kon02,kul04,coy04,lan07,sch09b,pre10}.
%
Under this scenario, we asked:
how rapidly does postzygotic isolation evolve?
what is the strength of genetic incompatibilities?
and what are the relative contributions
of compensatory and deleterious mutations to isolation?



Using an artificial life system,
we found that postzygotic isolation
due to compensation was rapid:
hybrids between compensated populations
had significantly lower fitness
than hybrids between populations that did not undergo compensation,
regardless of the effect size
of the initial deleterious mutations.
%
We also found that compensatory adaptation
formed stronger genetic incompatibilities:
the rate at which hybrid fitness
decayed with the number of mutations
was significantly greater
for hybrids between compensated populations.
%
Finally, we found that both
deleterious and compensatory mutations
contributed to postzygotic isolation:
hybrids with different proportions
of compensatory and deleterious mutations
were unfit unless all or none
of both types of mutations were present.



Two important implications to the genetics of postzygotic isolation
can be drawn from our findings.
%
First, evidence of genotypic diversification between species
may not correlate with phenotypic diversification.
%
This is because compensatory adaptation can build up genetic differences
without altering phenotypic characteristics.
%
In fact, compensatory adaptation may act under stabilizing selection
to maintain phenotypes despite continual accumulation of
deleterious mutations \cite{har96}.
%
Second, evidence of positive selection
at loci that contribute to speciation (`speciation genes')
may not always be the result of diversification
due to ecological adaptation
but instead be the footprint of compensatory adaptation.
%
Our results corroborate the view that
adaptation to the internal, genetic environment
may be important in the development
of intrinsic postzygotic isolation \cite{pre10,pha09}.



In describing his mechanism of founder speciation, Ernst Mayr stated that
``\dots\ the mere change of the genetic environment may change the
selective value of a gene very considerably'' \cite{tem08}.
%
In cases where founder events cause the fixation or high frequency
of deleterious mutations, compensatory adaptation may provide a
mechanism by which genes with altered selective values change
in allele frequency.
%
Furthermore, Templeton's model of genetic transilience recognizes
that founder populations may be affected by genetic drift
while new mutations and recombination increase genetic variation \cite{tem08}.
%
Genetic drift alone can raise the frequency of deleterious mutations
in a founder population while increased genetic variation
may introduce new compensatory mutations on which selection can act.
%
Our study using experimental evolution with populations of digital organisms
suggests that such compensatory mutations can rapidly generate
postzygotic barriers, and therefore supports a novel genetic mechanism
for models of founder or peripatric speciation.
% Check out paper:
% Increased mitochondrial mutation frequency after an island colonization: positive selection or accumulation of slightly deleterious mutations?


Our study took advantage
of the recent implementation of sexual reproduction
in the artificial life platform Avida \cite{mis06},
and extended its application
as proof of principle to the field of speciation.
%
We demonstrated that Avida is a useful tool to complement
other approaches in speciation research as it allows for
the direct observation of evolution and reproductive isolation in action.
%
Furthermore, our conclusions will motivate additional research
into compensatory adaptation as a viable mechanism for speciation,
to be further explored in biological systems.
%
The notion that organisms construct or choose
their own microhabitats \cite{lew00}
suggests that organisms are somewhat
resilient to changes in the external environment.
%
This reduced emphasis on the external environment in evolution
is supported by our conclusion that environmental differences
between allopatric populations are not essential for genetic diversification.


\section*{Materials and Methods}

\subsection*{Avida}

Experiments with digital organisms were carried out using Avida (ver.\ 2.9.0),
freely available at http://avida.devosoft.org.
%
In Avida, digital organisms consist of a sequence of instructions (or `genome')
that encodes their ability to replicate and perform computational functions.
%
Sexual reproduction between two organisms exchanges
a random region of their genomes,
then places one or two offspring back into the population \cite{mis06}.
%
Organisms are rewarded with a higher replication rate if they evolve
the ability to perform computational functions,
each of which requires a specific but not unique sequence of instructions.
%
Variation in the efficiency of replication and in the ability
to perform functions arises via recombination and mutation.
%
Because faster replicators in a finite population leave
more copies of themselves,
digital organisms evolve via natural selection
and genetic drift (i.e., evolution is not simulated).
%
Fitness, an estimate of an organism's reproductive rate,
is measured as the amount of rewards obtained divided
by the number of steps required to replicate.
%
Although there are many configurable settings in Avida (e.g., mutation rate),
processes like natural selection and epistasis occur spontaneously.



\section*{Strains and experimental conditions}

The `default' digital organism could replicate
but could not perform computational functions,
and its genome length was 80 instructions.
%
The population size was set to 10,000 organisms,
and the copy mutation probability was set to 0.0005 per instruction
(i.e., 0.04 per genome per generation).
%
Other types of mutations, such as insertions or duplications,
were not permitted because they may disrupt homologous recombination.
%
Digital organisms were configured to reproduce sexually,
and the two recombinant offspring were set to replace random individuals
in the population when no free space was available.
%
The `diverse' environment rewarded the nine computational functions
commonly used in Avida experiments \cite{len03}.
%
The small genome size of digital organisms caused reversions
to be common during compensatory adaptation in preliminary runs.
%
To guarantee that genotypic changes
were due to mutations at secondary loci%
---as has been observed in biological organisms \cite{bur99,est11}---%
we prevented reversions from occurring.



\section*{Isolation of mutants}

For each treatment, we generated sets of 10,000 random mutant genotypes
until either four mutants with the desired number of mutations and
mutational effect size were found, or until 100 million mutants
had been searched.
%
For small deleterious mutations, effect sizes of
0.01, 0.02, 0.03, 0.04, 0.05, 0.06, 0.07, 0.08, 0.09, and 0.1
(each with up to a 0.0049 deviation) were identified for genotypes carrying
1, 2, 3, 4, or 5 mutations.
%
For large deleterious mutations, mutants with effect sizes of
0.1, 0.2, 0.3, 0.4, 0.5, 0.6, 0.8, and 0.9 (with up to a 0.049 deviation)
for each mutation number of 1, 2, 3, 4, and 5 were identified.
%
(Effect size is the difference between
the fitnesses of the ancestral and the mutated genotype,
relative to the ancestral genotype.)
%
We isolated a total of 339 small-effect mutants
(after running the above procedure twice) and 133 large-effect mutants.
%
We implemented a different searching procedure for neutral mutations
because the probability of several random mutations resulting
in a neutral genotype was very low.
%
To find neutral mutants, we first generated 10,000 single-mutants
and selected the first one that was neutral
(i.e., relative fitness of exactly 1.0).
%
This procedure was repeated starting with the neutral mutant until
the desired number of mutations was reached or until the
recursion was exhausted.
%
We isolated a total of 74 neutral mutants.



\section*{Compensated populations}

We regarded a population as compensated if its most common genotype
(i) reached a fitness of at least 1.0 relative to its ancestor,
(ii) did not acquire mutations that were beneficial on their own, and
(iii) did not acquire mutations that were deleterious when they first appeared.
%
To determine condition (ii), the fitness effect of each
secondary mutation in the most common genotype of the population
was tested in the genetic background of the ancestor.
%
If any `transformant' had a relative fitness above 1.0,
then that population was not considered as compensated
because mutations were beneficial on their own
(i.e., generally beneficial, not compensatory).
%
To determine condition (iii), we sequentially examined the
most common genotype of the population every 10 updates ($\sim$2.4 generations).
%
If the gain of any mutation resulted in a lower fitness than the
genotype at the immediately preceding step (10 updates),
then that population was not considered as compensated.


\section*{Hybridization method}

Hybrids were created by the same method in which Avida
creates recombinants during sexual reproduction:
the genetic region between two crossover points
are exchanged between two parents to produce two offspring.
%
Hybridizations, whether between compensated populations
and their ancestor or between pairs of compensated populations,
involved creating every possible hybrid
between the most common genotype in each population.


\section*{Statistics}

To compare the fitness of hybrids among treatments (Fig. \ref{fig1}),
we determined wheth\-er their 95\% bootstrap confidence intervals overlapped.
%
For each treatment, bootstrap replicates were set to contain the
same number of samples per ancestor---established as the mean number
of observed samples per ancestor---in order to minimize any
potential bias from the fact that ancestors yielded different
numbers of compensated populations.
%
All bootstraps contained 10,000 replicates.
%
To determine the linear relationship between the number of mutations
in hybrids and their fitness (Fig. \ref{fig2}),
we fit a linear least squares model to each bootstrap replicate.
%
We compared the 95\% bootstrap confidence intervals of the slopes
among treatments to establish the relative effect of mutation number
on hybrid fitness.
%
Statistical analyses were performed in R (ver.\ 2.8.1).


\section*{Acknowledgments}

We thank R. E. Lenski, D. W. Schemske, and C. Ofria, I. Dworkin,
and A. Gerstein for helpful discussion and comments on the manuscripts.
%
This work was supported by the Quantitative
Biology Initiative at Michigan State University (MSU)
and the BEACON Center for the Study of Evolution in Action.
%
Computational experiments were made possible by the
High Performance Computing Center at MSU.
%
This material is based in part upon work supported
by the National Science Foundation under Cooperative Agreement No. DBI-0939454.
Any opinions, findings, and conclusions or recommendations
expressed in this material are those of the authors
and do not necessarily reflect the views of the National Science Foundation.

\bibliographystyle{plos2009}
\bibliography{comp_spp}

\pagebreak

\begin{figure}
\caption{Fitness of hybrids after 25,000 updates of parental evolution.
  Populations compensated for either small-effect or large-effect
  deleterious mutations.
  (A) Hybridizations between compensated genotypes and their ancestor.
  (B) Hybridizations between pairs of compensated genotypes
  sharing the same ancestor.
  `Del. Alone' include only the initial deleterious mutations
  before compensation,
  `Comp. Alone' include only the compensatory mutations after compensation,
  and `Del. + Comp.' include both deleterious and compensatory mutations.
  `Control' include all mutations accumulated neutrally.
  Error bars are 95\% bootstrap confidence intervals.}
\label{fig1}
\end{figure}



\begin{figure}
\caption{Strength of genetic incompatibilities in hybrids.
  (A) Least squares linear fit of the mean number of mutations in hybrids
  between ancestor and compensated genotypes (AC) against their mean fitness.
  (B) Same as (A) except for hybrids between pairs of compensated genotypes
  (CC). Insets: Means and 95\% bootstrap confidence intervals of each slope
  where shared letters indicate overlapping confidence intervals.}
\label{fig2}
\end{figure}



\begin{figure}
\caption{Fitness of hybrids with intermediate parental contributions
  of deleterious and compensatory mutations.
  (A) Mean fitness of hybrids between the ancestor
  and compensated parents that inherit the specified percent of small-effect
  deleterious and compensatory mutations. (B) Same as (A) except for
  large-effect deleterious mutations.}
\label{fig3}
\end{figure}



\begin{table}
\caption{Number of mutants and compensated populations per ancestor for
  the different treatments}
\begin{tabular}{lccc}
Treatment & Ancestor & Mutants & Compensated \\
\hline
        & 1 & 25 & - \\
Neutral & 2 & 25 & - \\
        & 3 & 24 & - \\
\hline
        & 1 & 71  & 10 \\
Small   & 2 & 199 & 15 \\
        & 3 & 69  & 6  \\
\hline
        & 1 & 44  & 9 \\
Large   & 2 & 44  & 5 \\
        & 3 & 45  & 6 \\
\hline
\end{tabular}
\label{tbl1}
\end{table}

\end{document}
